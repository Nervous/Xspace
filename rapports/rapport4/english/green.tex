\documentclass[10pt,a4paper]{article}
\usepackage[utf8]{inputenc}
\usepackage[T1]{fontenc}
\usepackage{wrapfig}
\usepackage{amsmath}
\usepackage{array}
\usepackage{graphicx}
\usepackage{tikz}

\title{Written part of the IT Project}
\author{Emmanuel Guet}
\date{12/06/12}

\begin{document}
\maketitle
	\par In a few words, I will try to summarize the best way I can how we made our project, Space Symphonia, and how we tried to lead it from a �basic� space shooter to a great space Shoot�Em All that works with music.
	\par First of all, I am the one who wanted to make a Shoot�Em All, but Antoine Pietri wanted the game to change, like really, with the music. After a brief reflection, we were all agreed that was a good idea, so we adopted it. And that's when the name of the game quickly came to us ; indeed �Space� is where the game takes place, and �Symphonia� refers to the fact that the game works with music. 
	\par So, at first, I have to say that we weren�t really organized on the projet, meaning even if in specifications we have written some stuffs, we didn�t feel to really follow it. But in fact, we quickly focused on each part we were interested in. Antoine Pietri was really interested in the music analysis, Antoine Vallee was interested in all stuffs like the menu, the level editor, handling scores, and so on, and finally I was interested in the mechanicals of the game. 
	\par Because I am here to give my point of view on how the project did this year, I will mainly talk about my part of the project, but I will also talk about other parts my friends did. And as I just said, my part was to work on the �basics� of the game, meaning all about the player (possibility to move, shoot, and switch between weapons), items in the game (ennemis that have to work properly, or bonuses that have to give the correct effect to the player), the progress of a level (make a system that will allow anyone to create a level, simply by writing what may spawn, giving it the time and the position in the screen), and also collisions, which are like � really � important in that type of game. This is kind of all what I have done during the first two presentations. I enjoyed this part because it was like, starting from scratch (even if I used XNA, so basically it is not really starting from scratch but anyway �). And for the last two presentations, I worked on the proper content of the game, meaning I created the new ennemis, new bonuses and new weapons.
	\par I have to say that I really enjoyed working on some real content like ennemis or weapons, I believe this was really an interesting part. But I also have to say that all didn�t go smooth during the year. Yes, sometimes we had some disputes, about how much we worked each of us for instance, or even on some ideas we had. And, finally, I must say that this project, one of the reason I chose EPITA, was more rewarding than what I expected, both in human and technical aspect. I know we will have to do a lot of others projects in the future, and I can not wait to do those.
\end{document}
